\chapter{DESAIN DAN IMPLEMENTASI}
\label{chap:desainimplementasi}

% Ubah bagian-bagian berikut dengan isi dari desain dan implementasi

\section{Deskripsi Sistem}
\label{sec:deskripsisistem}

Pada tahap ini, kursi roda otonom dikembangkan untuk dapat mengikuti manusia menggunakan algoritma deteksi berbasis YOLOv8. Sistem ini dirancang dengan tujuan utama meningkatkan mobilitas pengguna yang membutuhkan bantuan dalam bergerak. Sistem terdiri dari komponen perangkat keras dan perangkat lunak, yang terintegrasi untuk mencapai tujuan ini.

\subsection{Komponen Sistem}
\label{subsec:komponensistem}

Sistem ini terdiri dari:

\begin{enumerate}[nolistsep]
    \item \textbf{Anaconda Navigator}: Digunakan sebagai lingkungan pengembangan untuk menjalankan kode terkait YOLOv8 dan analisis lainnya.
    \item \textbf{Arduino IDE}: Mengembangkan dan mengunggah kode ke ESP32 yang mengendalikan motor dan sensor.
    \item \textbf{Laptop}: Berfungsi sebagai pusat pemrosesan data yang lebih kompleks dan pengembangan perangkat lunak.
    \item \textbf{Kamera (OV5640 5MP)}: Menangkap gambar lingkungan secara real-time dan mendeteksi target (manusia) menggunakan YOLOv8.
    \item \textbf{ESP32 Devkit V1}: Berfungsi sebagai pengontrol utama yang mengelola data dari kamera dan mengendalikan pergerakan kursi roda.
    \item \textbf{2 Kontroller Motor}: Mengendalikan motor DC yang menggerakkan kursi roda.
    \item \textbf{2 DC-DC Voltage Regulator}: Mengatur tegangan untuk komponen elektronik agar tetap stabil.
    \item \textbf{2 Motor DC}: Menggerakkan kursi roda, dikontrol melalui driver motor yang menerima sinyal dari mikroprosesor.
    \item \textbf{Baterai 24V}: Sebagai sumber daya utama untuk keseluruhan sistem, termasuk mikroprosesor, motor, dan perangkat lainnya.
\end{enumerate}

\subsection{Arsitektur Sistem}
\label{subsec:arsitektursistem}

Arsitektur sistem dirancang agar kamera menangkap gambar secara terus-menerus, kemudian mengirimkan data ke mikroprosesor ESP32 untuk diproses. YOLOv8 digunakan untuk mendeteksi manusia dan memberikan informasi koordinat posisi target. Informasi ini kemudian digunakan oleh mikroprosesor untuk mengontrol motor dan mengarahkan kursi roda agar dapat mengikuti pergerakan target secara dinamis.

\section{Perangkat Keras}
\label{sec:perangkathardware}

Desain perangkat keras melibatkan integrasi antara beberapa komponen yang disebutkan sebelumnya. Kamera dipasang di bagian depan kursi roda untuk mendapatkan sudut pandang optimal terhadap target. Mikroprosesor ESP32 ditempatkan di bagian bawah kursi roda bersama dengan driver motor dan baterai untuk menjaga keseimbangan.

\begin{itemize}[nolistsep]
    \item \textbf{Laptop}: Digunakan sebagai pusat pemrosesan utama untuk menjalankan YOLOv8 dan menganalisis hasil deteksi. Laptop ini juga berfungsi untuk mengembangkan perangkat lunak dan melatih model menggunakan Anaconda Navigator.
    \item \textbf{Kamera OV5640}: Kamera ini terhubung ke ESP32 untuk melakukan streaming video dengan resolusi lebih tinggi karena menggunakan sensor 5MP, yang memungkinkan pengenalan target lebih akurat. Posisi kamera memungkinkan pengenalan target secara optimal ketika pengguna bergerak.
    \item \textbf{ESP32 sebagai Pengontrol}: ESP32 bertindak sebagai otak dari kursi roda otonom. Perangkat ini menerima data dari kamera dan kemudian menjalankan algoritma YOLOv8 untuk mendeteksi target, lalu mengirim sinyal kontrol ke driver motor.
    \item \textbf{Driver Motor L298N}: Driver ini digunakan untuk mengontrol kecepatan dan arah putaran motor DC yang menggerakkan kursi roda.
\end{itemize}

\section{Perangkat Lunak}
\label{sec:perangkatlunak}

Perangkat lunak yang digunakan dalam sistem ini terdiri dari beberapa modul, yang meliputi:

\begin{itemize}[nolistsep]
    \item \textbf{Algoritma Deteksi YOLOv8}: YOLOv8 digunakan untuk mendeteksi manusia dalam gambar yang ditangkap oleh kamera. Model ini dilatih untuk mengenali bentuk dan posisi manusia sehingga kursi roda dapat mengikuti target dengan tepat.
    \item \textbf{Estimasi Pose MediaPipe}: MediaPipe digunakan untuk mendeteksi landmark tubuh manusia setelah objek terdeteksi menggunakan YOLOv8. Framework ini dipilih karena kemampuannya dalam mendeteksi keypoints pada tubuh manusia dengan akurasi tinggi, bahkan dalam kondisi pencahayaan yang beragam.
    \item \textbf{Komunikasi Data}: Sistem menggunakan protokol MQTT untuk mengirimkan data antara ESP32 yang terhubung ke kamera dan unit kontrol utama.
    \item \textbf{Kontrol Motor}: Modul kontrol motor diimplementasikan pada ESP32 untuk mengatur arah dan kecepatan motor berdasarkan posisi target yang terdeteksi.
\end{itemize}

\subsection{Dataset Citra}
\label{subsec:datasetcitra}

Dataset citra yang digunakan dalam penelitian ini terdiri dari kumpulan gambar yang diambil menggunakan kamera OV5640. Objek yang dideteksi pada gambar ini adalah manusia, yang difungsikan sebagai target untuk diikuti oleh kursi roda. Dataset mencakup berbagai pose dan posisi manusia, serta kondisi pencahayaan yang berbeda, agar model YOLOv8 dapat dilatih secara optimal dalam mengenali target dalam berbagai situasi. Gambar-gambar ini diperoleh dari setiap frame video yang diambil menggunakan kamera yang terhubung dengan komputer. Proses pendeteksian dilakukan secara real-time untuk memastikan sistem mampu mengidentifikasi target secara berkelanjutan.

\subsection{Labeling}
\label{subsec:labeling}

Dataset citra yang telah diperoleh kemudian melalui proses labeling dan augmentasi menggunakan alat seperti Roboflow, yang menyediakan berbagai fitur untuk mempermudah proses ini. Proses labeling ini melibatkan impor dataset, pemberian label pada setiap gambar, dan augmentasi data. Labeling dilakukan untuk memastikan bahwa setiap objek dalam gambar dikenali dengan benar, terutama manusia yang menjadi target. Penamaan class harus konsisten dengan objek yang akan dideteksi untuk memastikan model YOLOv8 dapat dilatih dengan baik.

Selain itu, Roboflow juga menyediakan fitur preprocessing dataset yang membantu menstandarkan format gambar, misalnya mengubah ukuran semua gambar menjadi seragam. Langkah ini penting untuk menjaga konsistensi dataset sebelum melatih model. Beberapa fitur yang tersedia antara lain Auto-Orient, Resize, Grayscale, Auto Adjust Contrast, Isolate Objects, Static Crop, Tile, Modify Classes, dan Filter Null. Preprocessing ini memastikan bahwa data yang digunakan memiliki kualitas yang seragam untuk meningkatkan performa model.

Fitur augmentasi data juga disediakan untuk menambah variasi pada dataset. Augmentasi bertujuan untuk meningkatkan keragaman dalam dataset, yang pada akhirnya membantu meningkatkan performa model dalam mendeteksi manusia. Proses augmentasi ini sangat penting dalam tugas akhir ini karena membutuhkan variasi yang luas dalam data citra manusia, yang memungkinkan sistem lebih robust dalam berbagai kondisi.

\subsection{Klasifikasi YOLOv8}
\label{subsec:klasifikasiyolov8}

Dalam proses klasifikasi, setiap citra yang telah melalui proses labeling dikenali dengan menggunakan YOLOv8 yang telah dilatih untuk mendeteksi manusia pada citra. Model ini mampu mendeteksi keberadaan manusia secara akurat dan real-time, memungkinkan sistem untuk mengidentifikasi target dengan efisien. Setiap citra diproses oleh model YOLOv8, yang kemudian memberikan output berupa bounding box dan confidence score yang menunjukkan keberadaan manusia serta keyakinan model terhadap deteksi tersebut. Proses ini dilakukan secara real-time, memungkinkan sistem untuk memberikan umpan balik langsung ke kontrol kursi roda berdasarkan deteksi manusia dalam gambar.

Model YOLOv8 yang digunakan memiliki output kelas utama yaitu manusia. Model ini menghasilkan bounding box, yang menunjukkan area lokasi objek, dan confidence score untuk memberikan keyakinan deteksi. Algoritma YOLOv8 menggunakan Convolutional Neural Network (CNN) sebagai basisnya, yang berfungsi untuk mengekstraksi fitur dari gambar input, lalu memprediksi bounding box dan kelas objek langsung dari gambar tersebut. Proses ini diilustrasikan pada Gambar, yang menunjukkan blok diagram arsitektur model YOLOv8.

YOLOv8 terdiri dari beberapa lapisan, antara lain Conv2D (Convolutional 2D), Blok C2f, Blok SPPF (Spatial Pyramid Pooling Fast), lapisan Upsampling, dan lapisan Concatenate. Setiap lapisan ini memiliki peran tertentu dalam memproses gambar, mengekstraksi fitur, dan membuat prediksi yang tepat. Lapisan Conv2D digunakan untuk mengekstraksi fitur dari gambar input, sementara Blok C2f dan SPPF membantu dalam pemrosesan informasi dari berbagai skala untuk menangkap detail yang lebih baik. Lapisan Upsampling digunakan untuk meningkatkan resolusi fitur, sementara lapisan Concatenate menggabungkan berbagai informasi dari jalur yang berbeda untuk membuat representasi yang lebih kuat.

Gambar menunjukkan setiap jenis lapisan dalam YOLOv8 dengan warna yang berbeda, seperti lapisan Conv2D yang ditampilkan dalam warna biru, Blok C2f dengan warna kuning, lapisan SPPF dengan warna hijau muda, lapisan Upsampling dengan warna merah muda, dan lapisan Concatenate dengan warna ungu. Visualisasi ini membantu memahami bagaimana model YOLOv8 memproses gambar dan bagaimana setiap lapisan berkontribusi dalam menghasilkan bounding box dan prediksi kelas yang akurat. Selain itu, input dan output dari setiap lapisan dapat dilihat pada Gambar, yang menggambarkan detail dari tiap proses pemrosesan dalam model YOLOv8.

Lapisan deteksi pada YOLOv8 menghasilkan output yang mencakup bounding box dan confidence score untuk manusia yang terdeteksi, yang kemudian digunakan sebagai acuan dalam sistem untuk mengikuti target. Model ini dioptimalkan untuk dapat mengenali manusia dengan berbagai posisi dan kondisi pencahayaan, sehingga kursi roda dapat beradaptasi dengan pergerakan target dalam berbagai situasi secara efektif.

\subsection{Estimasi Pose MediaPipe}
\label{subsec:estimasi_pose_mediapipe}

Pada penelitian ini, pose manusia dideteksi menggunakan Python dengan library OpenCV dan MediaPipe. Proses dimulai setelah objek manusia terdeteksi dalam frame, di mana MediaPipe kemudian digunakan untuk mengidentifikasi landmark pada tubuh. MediaPipe dipilih karena kemampuannya mendeteksi titik kunci (keypoints) pada tubuh manusia dengan akurasi tinggi dalam berbagai kondisi pencahayaan dan posisi. Setelah landmark berhasil dideteksi, titik-titik yang relevan akan digambarkan menggunakan garis untuk membentuk kerangka sesuai dengan pose tubuh.

Proses deteksi diawali dengan inisialisasi kamera yang menangkap frame secara real-time. Setelah frame diterima, dilakukan pra-pengolahan seperti konversi gambar ke skala abu-abu untuk mengurangi kompleksitas dan meningkatkan kecepatan deteksi. Gambar yang telah dipra-pengolah tersebut kemudian diproses oleh model MediaPipe untuk mendeteksi pose.

Dalam penelitian ini, beberapa landmark yang dipilih untuk analisis adalah titik-titik pada siku, lengan bawah, dan bahu kanan serta kiri. Pemilihan titik-titik ini didasarkan pada visibilitas dan konsistensi dalam proses deteksi. Tabel menampilkan nomor dan nama keypoint yang digunakan dalam estimasi pose:

\begin{table}[H]
\centering
\begin{tabular}{|c|c|}
\hline
Nomor Keypoint & Nama Keypoint \\
\hline
11 & RIGHT SHOULDER \\
12 & LEFT SHOULDER \\
14 & RIGHT ELBOW \\
16 & RIGHT WRIST \\
\hline
\end{tabular}
\caption{Tabel Keypoint yang digunakan}
\label{tab:keypoints}
\end{table}

Setelah landmark diperoleh, jarak antar titik pada piksel dihitung. Pengukuran ini dilakukan dengan menggunakan jarak Euclidean antara dua titik kunci, yang merupakan metode efisien untuk menghitung jarak dalam ruang dua dimensi. Nilai jarak ini kemudian digunakan sebagai acuan untuk pergerakan kursi roda mengikuti target di depan.

\subsection{Pemrosesan Citra}
\label{subsec:pemrosesan_citra}

Pemrosesan citra dilakukan untuk meningkatkan kualitas gambar yang diambil dari kamera dan mempermudah deteksi objek. Langkah-langkah pemrosesan citra meliputi konversi warna, penghilangan noise, dan pemfilteran tepi untuk menyorot bagian-bagian penting dari gambar.

\subsubsection{3.3.5.1 Pembuatan Tracking ID untuk Mengunci Target}
\label{subsubsec:tracking_id}

Sistem menggunakan YOLOv8 untuk mendeteksi target manusia dalam frame, dan kemudian memberikan ID unik pada setiap objek yang terdeteksi. Tracking ID ini digunakan untuk mengidentifikasi dan melacak individu yang sama pada frame berikutnya, sehingga kursi roda dapat terus mengikuti target meskipun ada pergerakan dalam frame. Algoritma ini membantu kursi roda untuk tetap fokus pada target yang sama selama proses pelacakan.

\subsubsection{3.3.5.2 Keputusan untuk Bergerak Maju}
\label{subsubsec:keputusan_bergerak_maju}

Setelah target terdeteksi dan tracking ID dibuat, sistem akan menghitung jarak antara kursi roda dengan target. Jika jarak target berada dalam kisaran yang aman dan berada di tengah frame, sistem akan mengirimkan perintah untuk bergerak maju. Hal ini dilakukan dengan menggunakan algoritma yang menghitung jarak dari bounding box target dan memastikan bahwa target berada pada posisi yang tepat untuk diikuti.

\subsubsection{3.3.5.3 Keputusan untuk Berbelok ke Kiri}
\label{subsubsec:keputusan_belok_kiri}

Jika target bergerak ke arah kiri dan keluar dari batas tengah frame, kursi roda akan mengirimkan perintah untuk berbelok ke kiri. Sistem mendeteksi posisi target di bagian kiri frame dan memberikan instruksi ke motor untuk mengarahkan kursi roda ke kiri agar tetap dapat mengikuti pergerakan target secara optimal.

\subsubsection{3.3.5.4 Keputusan untuk Berbelok ke Kanan}
\label{subsubsec:keputusan_belok_kanan}

Sebaliknya, jika target bergerak ke arah kanan dan keluar dari batas tengah frame, sistem akan mengirimkan perintah untuk berbelok ke kanan. Posisi target yang berada di sisi kanan frame akan memicu motor untuk berbelok ke kanan, memastikan kursi roda dapat menyesuaikan posisinya dengan target yang bergerak.

\subsubsection{3.3.5.5 Keputusan Jika Target Menghilang dari Frame}
\label{subsubsec:keputusan_target_menghilang}

Apabila target menghilang dari frame, sistem akan mencari target berdasarkan posisi terakhir yang terdeteksi. Jika target tidak ditemukan dalam jangka waktu tertentu, kursi roda akan berhenti untuk menghindari pergerakan yang tidak diinginkan. Namun, jika ada target baru yang terdeteksi di frame, sistem akan membuat tracking ID baru dan melanjutkan pelacakan terhadap target tersebut. Sistem akan menunggu hingga target muncul kembali dalam frame atau mendeteksi target baru untuk melanjutkan pelacakan. Pendekatan ini memastikan bahwa kursi roda tidak bergerak tanpa arah yang jelas ketika target tidak terlihat.

\subsection{Komunikasi Data}
\label{subsec:komunikasi_data}

Pada sistem ini, komunikasi data dilakukan menggunakan protokol MQTT. MQTT adalah protokol yang ringan dan ideal untuk aplikasi yang membutuhkan pengiriman data secara efisien dan real-time. Dalam proyek ini, MQTT digunakan untuk mengirimkan data antara ESP32 yang terhubung dengan kamera dan unit kontrol utama yang bertanggung jawab atas analisis data dan kontrol motor. Protokol ini memungkinkan ESP32 dan unit kontrol utama untuk bertukar informasi secara cepat, yang sangat penting dalam memastikan bahwa kursi roda dapat bereaksi terhadap perubahan posisi target dengan segera.

Kamera yang terhubung ke ESP32 menangkap gambar secara real-time dan mengirimkannya ke unit kontrol melalui server WiFi. Data dari kamera, setelah diolah oleh YOLOv8, dikirim ke unit kontrol untuk analisis lebih lanjut melalui MQTT. ESP32 lainnya, yang berfungsi sebagai pengontrol motor, juga menerima instruksi melalui MQTT untuk menentukan pergerakan yang sesuai. Dengan menggunakan MQTT, sistem dapat memelihara komunikasi yang cepat dan stabil, memastikan bahwa seluruh komponen bekerja secara sinkron.

\subsection{Kontrol Motor}
\label{subsec:kontrol_motor}

Kontrol motor pada sistem kursi roda ini diimplementasikan menggunakan ESP32 yang bertindak sebagai pengendali utama untuk arah dan kecepatan motor. Setelah target manusia terdeteksi oleh kamera dan diproses oleh YOLOv8, informasi mengenai posisi target dikirim ke ESP32 untuk menentukan perintah pergerakan yang sesuai. Berdasarkan posisi target di dalam frame, ESP32 akan memberikan sinyal kepada driver motor untuk mengendalikan arah gerakan kursi roda, baik bergerak maju, berbelok ke kiri, berbelok ke kanan, atau berhenti.

Motor dikendalikan melalui sinyal PWM (Pulse Width Modulation) yang dihasilkan oleh ESP32, yang digunakan untuk mengatur kecepatan motor secara halus. Kombinasi sinyal arah dan PWM memungkinkan sistem untuk menggerakkan motor dengan responsif, baik untuk mengejar target, berbelok, atau berhenti secara tiba-tiba jika diperlukan. Selain itu, sistem juga menggunakan kontrol loop untuk terus memantau posisi target dan menyesuaikan pergerakan kursi roda secara real-time agar tetap dapat mengikuti target dengan akurat. Proses kontrol ini berlangsung terus-menerus untuk memastikan bahwa kursi roda selalu dapat beradaptasi dengan pergerakan target.

\subsection{Kode Program}
\label{subsec:kode_program}

Kode program yang digunakan dalam penelitian ini dimulai dengan inisialisasi variabel dan perangkat keras yang diperlukan, termasuk mengaktifkan kamera untuk menangkap gambar secara real-time. Setelah gambar diperoleh, algoritma YOLOv8 digunakan untuk mendeteksi keberadaan manusia dalam frame tersebut. Jika manusia terdeteksi, program melanjutkan dengan mengidentifikasi bounding box dan memberi label pada objek, yang kemudian digunakan untuk melacak objek tersebut di frame berikutnya. Setelah itu, framework MediaPipe diimplementasikan untuk mendeteksi landmark tubuh, seperti bahu, siku, dan pergelangan tangan, yang memungkinkan sistem memperoleh pose manusia secara lebih rinci.

\begin{figure}[!ht]
  \centering
  \resizebox{0.9\linewidth}{!}{
    \begin{tikzpicture}[node distance=2cm]
      \node (start) [startstop] {Mulai};
      \node (initVars) [process, below of=start] {Inisialisasi Variabel dan Hardware};
      \node (captureFrame) [io, below of=initVars] {Ambil Frame dari Kamera};
      \node (yoloDetect) [process, below of=captureFrame] {Deteksi Objek Menggunakan Yolo};
      \node (A) [connector, below of=yoloDetect] {A};
      \node (C0) [connector, left of=initVars, xshift=-1.5cm, yshift=-1cm] {C};
  
      \node (A0) [connector, right of=start, xshift=4.5cm] {A};
      \node (checkBoxes) [decision, below of=A0] {Deteksi Objek?};
      \node (processBox) [process, below of=checkBoxes] {Proses Bounding Box};
      \node (trackingID) [process, below of=processBox] {Proses Tracking ID};
      \node (mediapipeProcessing) [process, below of=trackingID] {Deteksi Landmark Pose pada MediaPipe};
      \node (B) [connector, below of=mediapipeProcessing, xshift=1.5cm] {B};
  
      \node (B0) [connector, right of=A0, xshift=3.5cm] {B};
      \node (calcDirection) [process, below of=B0] {Proses Kalkulasi Arah dan Jarak};
      \node (controlWheelchair) [io, below of=calcDirection] {Kirim Data ke Kursi Roda};
      \node (display) [io, below of=controlWheelchair] {Gambar Arah pada Frame};
      \node (endLoop) [decision, below of=display, yshift=-.5cm] {Tombol 'q' Ditekan?};
      \node (C) [connector, right of=endLoop, xshift=1cm, yshift=-1.5cm] {C};
      \node (end) [startstop, below of=endLoop, yshift=-1cm] {Selesai};
  
      \node (searchPerson) [process, below of=mediapipeProcessing, xshift=-3cm] {Memuat Arah dan Jarak Sebelumnya};
      
      \draw [arrow] (start) -- (initVars);
      \draw [arrow] (initVars) -- (captureFrame);
      \draw [arrow] (captureFrame) -- (yoloDetect);
      \draw [arrow] (yoloDetect) -- (A);
      \draw [arrow] (C0) |- (captureFrame);
  
      \draw [arrow] (A0) -- (checkBoxes);
      \draw [arrow] (checkBoxes) -- node[anchor=west] {Ya} (processBox);
      \draw [arrow] (processBox) -- (trackingID);
      \draw [arrow] (trackingID) -- (mediapipeProcessing);
      \draw [arrow] (mediapipeProcessing) -- +(0,-2cm);
  
      \draw [arrow] (B0) -- (calcDirection);
      \draw [arrow] (calcDirection) -- (controlWheelchair);
      \draw [arrow] (controlWheelchair) -- (display);
      \draw [arrow] (display) -- (endLoop);
      \draw [arrow] (endLoop) -| node[anchor=south east] {Tidak} (C);
      \draw [arrow] (endLoop) -- node[anchor=west] {Ya} (end);
      
      \draw [arrow] (checkBoxes) -| node[anchor=south west] {Tidak} (searchPerson);
      \draw [arrow] (searchPerson) -- (B);
    \end{tikzpicture}
  }
  \caption{Flowchart program python}
\end{figure}

Program kemudian menghitung arah dan jarak target, yang digunakan untuk menentukan instruksi pergerakan kursi roda. Instruksi tersebut dikirimkan ke ESP32, yang mengendalikan motor kursi roda untuk mengikuti manusia secara otomatis. Selain itu, arah pergerakan digambarkan pada frame video yang ditampilkan sebagai bentuk visualisasi dari proses yang sedang berjalan. Program akan terus melakukan loop, menangkap gambar baru, memproses deteksi, dan mengirimkan perintah hingga pengguna menekan tombol 'q' untuk mengakhiri program.

\begin{figure}[!ht]
  \centering
  \resizebox{0.9\linewidth}{!}{
  \begin{tikzpicture}[node distance=2cm]

    \node (start) [startstop] {Mulai};
    \node (getCurrentTime) [io, below of=start] {Ambil Waktu\\Saat ini};
    \node (checkSocket) [decision, below of=getCurrentTime, yshift=-.5cm] {Socket Tersedia?};
    \node (checkData) [decision, below of=checkSocket, yshift=-1cm] {Arah Tersedia?};
    \node (A) [connector, below of=checkData, xshift=-2.5cm] {A};
    \node (B) [connector, below of=checkData, xshift=2.5cm] {B};

    \node (A0) [connector, right of=start, xshift=5.5cm] {A};
    \node (setTime) [process, below of=A0] {Penanda berada dalam \emph{delay} yang ditentukan};
    \node (compareData) [decision, below of=setTime, yshift=-.5cm] {Arah Berubah?};
    \node (sendC) [io, below of=compareData, xshift=2.5cm] {Kirim 'C\textbackslash n' \\ke Socket};
    \node (setSentFalse) [process, below of=sendC] {Penanda Belum dikirim};
    \node (C) [connector, below of=setSentFalse, xshift=-5cm, yshift=1cm] {C};

    \node (B0) [connector, right of=A0, xshift=3cm] {B};

    \node (C0) [connector, right of=B0, xshift=3cm] {C};
    \node (setData) [process, below of=C0] {Simpan Arah untuk referensi selanjutnya};
    \node (checkInterval) [decision, below of=setData, yshift=-.5cm] {Sudah dikirim \(\lor\) \emph{delay}?};
    \node (sendData) [io, below of=checkInterval, xshift=3cm] {Kirim Arah \\ke Socket};
    \node (setSentTrue) [process, below of=sendData] {Penanda \\Sudah dikirim};
    \node (end) [startstop, below of=setSentTrue, xshift=-6cm, yshift=-1cm] {Selesai};
    
    \draw [arrow] (start) -- (getCurrentTime);
    \draw [arrow] (getCurrentTime) -- (checkSocket);
    \draw [arrow] (checkSocket) -- node[anchor=east] {Ya} (checkData);
    \draw [arrow] (checkData) -| node[anchor=north west] {Ya} (A);
    \draw [arrow] (checkData) -| node[anchor=north east] {Tidak} (B);

    \draw [arrow] (A0) -- (setTime);
    \draw [arrow] (setTime) -- (compareData);
    \draw [arrow] (compareData) -| node[anchor=north east] {Ya} (sendC);
    \draw [arrow] (sendC) -- (setSentFalse);
    \draw [arrow] (compareData) -| node[anchor=north west] {Tidak} (C);
    \draw [arrow] (setSentFalse.west) -- +(-2.5cm,0);

    \draw [arrow] (C0) -- (setData);
    \draw [arrow] (setData) -- (checkInterval);
    \draw [arrow] (checkInterval) -| node[anchor=north east] {Tidak} (sendData);
    \draw [arrow] (checkInterval) -| node[anchor=north west] {Ya} (end);
    \draw [arrow] (sendData) -- (setSentTrue);
    \draw [arrow] (setSentTrue.south) |- +(-6cm,-1.5cm);

    \draw [arrow] (B0) |- +(5cm,-3cm);
    \draw [arrow] (checkSocket.east) -| node[anchor=north east] {Tidak} +(2cm,0) |-  +(12.7cm,-6.05cm);
  
  \end{tikzpicture}
  }
  \caption{Flowchart regulasi arah}
\end{figure}

Regulasi arah kursi roda dilakukan melalui pengiriman data arah ke socket yang berhubungan dengan ESP32. Program mengecek apakah socket tersedia dan apakah data arah berubah, serta menetapkan penanda untuk menghindari pengiriman data yang berulang. Sebelum arah diubah, data 'C\textbackslash n' dikirim terlebih dahulu untuk memastikan kursi roda berhenti dan stabil sebelum menerima instruksi arah baru. Hal ini penting untuk mencegah gerakan yang tidak diinginkan atau perubahan arah yang terlalu mendadak. Setiap kali arah berubah setelah berhenti, data baru akan dikirim ke socket untuk mengontrol motor kursi roda.

\begin{figure}[t]
  \centering
  \resizebox{0.9\linewidth}{!}{
  \begin{tikzpicture}[node distance=2cm]
    % Nodes
    \node (start) [startstop] {Mulai};
    \node (init) [process, below of=start] {Inisialisasi Arduino, WiFi, PWM dan Pin};
    \node (connect) [process, below of=init] {Menyambungkan Serial WiFi dan Server};
    \node (setup) [process, below of=connect] {Setup PinMode, ledc-Setup, ledcAttachPin};
    \node (A) [connector, below of=setup, xshift=2cm, yshift=-.5cm] {A};

    \node (A0) [connector, right of=start, xshift=3cm] {A};
    \node (connected?) [decision, below of=A0, text width=2cm] {Perangkat Terhubung?};
    \node (msg?) [decision, below of=connected?, yshift=-.5cm] {Pesan Diterima?};

    \node (stop) [startstop, right of=connected?, xshift=3cm] {Stop};
    \node (readstr) [process, right of=msg?, xshift=3cm] {Read String sampai terdapat '\textbackslash n'};
    \node (extract) [process, below of=readstr] {Ekstrak String menjadi Arah dan Kecepatan};
    \node (move) [io, below of=msg?, text width=3.5cm] {Menggerakkan Motor Kursi Roda};


    % Arrows
    \draw [arrow] (start) -- (init);
    \draw [arrow] (init) -- (connect);
    \draw [arrow] (connect) -- (setup);
    \draw [arrow] (setup) |- ++(0,-1.55cm) -| (A);

    \draw [arrow] (A0) -- (connected?);
    \draw [arrow] (connected?) -- node[anchor=east] {Ya} (msg?);
    \draw [arrow] (connected?) -- node[anchor=south] {Tidak} (stop);
    \draw [arrow] (msg?) -- node[anchor=south] {Ya} (readstr);
    \draw [arrow] (msg?.west) -| node[anchor=north west] {Tidak} ++(-.5,0) |- (connected?);
    \draw [arrow] (readstr) -- (extract);
    \draw [arrow] (extract) -- (move);
    \draw [arrow] (move.south) |- ++(0,-.5) -| (A);
  \end{tikzpicture}
  }
  \caption{Flowchart ESP B}
\end{figure}

Untuk ESP32 CAM, sistem dimulai dengan inisialisasi kamera dan pengaturan WiFi sebagai Access Point. Server WiFi kemudian dijalankan pada port 80 dan 81 untuk melakukan streaming data secara real-time. Setelah konfigurasi ini, ESP-NOW digunakan untuk komunikasi antara kedua ESP32, yang memungkinkan transmisi data tanpa harus terhubung ke jaringan WiFi eksternal.

Frame yang ditangkap oleh kamera dikirim melalui alamat IP tertentu, yaitu `http://192.168.4.1:81/stream`, yang juga digunakan dalam kode program Python untuk melakukan streaming data. Data arah yang diterima dari client kemudian diteruskan ke ESP32 B melalui ESP-NOW untuk mengontrol kursi roda. Sistem ini dirancang untuk memastikan komunikasi yang cepat dan handal antara ESP32 A dan ESP32 B, sehingga kursi roda dapat merespons pergerakan target dengan tepat.

\begin{figure}[!ht]
  \centering
  \resizebox{0.9\linewidth}{!}{
  \begin{tikzpicture}[node distance=2cm]
    \node (start) [startstop] {Mulai};
    \node (camera) [process, below of=start] {Inisialisasi Kamera};
    \node (wifi) [process, below of=camera] {Set Wi-Fi sebagai Access Point dan Station};
    \node (server) [process, below of=wifi] {Mulai Wifi Server di port 80 dan 81};
    \node (espNow) [process, below of=server] {Inisialisasi ESP-NOW dengan Peer ESP B};
    \node (A) [connector, below of=espNow] {A};

    \node (A0) [connector, right of=start, xshift=4cm] {A};
    \node (stream) [decision, below of=A0, text width=2cm] {Stream Terhubung?};
    \node (peer) [io, below of=stream, text width=4.5cm] {Stream frame via 192.168.4.1:81/stream};
    \node (client) [decision, below of=peer, text width=2cm] {Client Terhubung?};
    \node (readData) [process, right of=client, xshift=3cm] {Baca Data dari Client di port 80};
    \node (sendESPNow) [io, below of=readData, text width=3cm] {Kirim Data via ESP-NOW};
    \node (clientStop) [process, below of=client] {Client Terputus};
    \node (A1) [connector, below of=clientStop] {A};

    \node (stop) [startstop, right of=stream, xshift=3cm] {Selesai};

    % Arrows
    \draw [arrow] (start) -- (camera);
    \draw [arrow] (camera) -- (wifi);
    \draw [arrow] (wifi) -- (server);
    \draw [arrow] (server) -- (espNow);
    \draw [arrow] (espNow) -- (A);

    \draw [arrow] (A0) -- (stream);
    \draw [arrow] (stream) -- (peer);
    \draw [arrow] (peer) -- (client);

    \draw [arrow] (client) -- node[anchor=north] {Ya} (readData);
    \draw [arrow] (client) -- node[anchor=east] {Tidak} (clientStop);
    \draw [arrow] (readData) -- (sendESPNow);
    \draw [arrow] (sendESPNow) -- (clientStop);
    \draw [arrow] (clientStop) -- (A1);
  \end{tikzpicture}
  }
  \caption{Flowchart ESP A}
\end{figure}

Pada sisi ESP32 Motor, perangkat dimulai dengan inisialisasi berbagai komponen, seperti PWM, WiFi, dan pin untuk mengendalikan motor. Setelah terhubung ke WiFi, ESP32 B menunggu pesan dari server yang berfungsi sebagai pusat kendali. Jika pesan diterima, string yang mengandung informasi arah dan kecepatan akan diekstrak, kemudian diolah untuk menentukan pergerakan motor kursi roda. Sistem menggunakan data ini untuk mengatur arah dan kecepatan motor, yang memastikan kursi roda dapat mengikuti target secara akurat dan responsif. Proses ini dilakukan secara berulang untuk setiap update data yang diterima, memungkinkan kursi roda menyesuaikan gerakan dengan perubahan posisi target secara real-time.




% Contoh pembuatan potongan kode
\begin{lstlisting}[
  language=C++,
  caption={Program halo dunia.},
  label={lst:halodunia}
]
#include <iostream>

int main() {
    std::cout << "Halo Dunia!";
    return 0;
}
\end{lstlisting}

\lipsum[2-3]

% Contoh input potongan kode dari file
\lstinputlisting[
  language=Python,
  caption={Program perhitungan bilangan prima.},
  label={lst:bilanganprima}
]{program/bilangan-prima.py}

\lipsum[4]
