\chapter{PENDAHULUAN}
\label{chap:pendahuluan}

% Ubah bagian-bagian berikut dengan isi dari pendahuluan

\section{Latar Belakang}
\label{sec:latarbelakang}

Navigasi kursiroda telah mengalami perkembangan pesat, dan sistem autopilot menjadi fokus utama dalam meningkatkan kenyamanan dan keamanan pengguna. Deteksi pose manusia memainkan peran krusial dalam mengidentifikasi posisi dan gerakan pengguna. YOLOv8n-pose, sebagai model deteksi pose yang lebih canggih, menawarkan potensi untuk meningkatkan akurasi dan efisiensi dalam pemantauan manusia pada kursiroda.
Mobilitas merupakan aspek penting dalam kehidupan manusia, yang memungkinkan individu untuk beraktivitas, berinteraksi dengan lingkungan, dan mencapai kemandirian. Kursi roda telah menjadi alat bantu mobilitas yang krusial bagi jutaan orang di seluruh dunia, terutama bagi mereka yang mengalami keterbatasan fisik.

Namun, memastikan keamanan dan kemandirian pengguna kursi roda, terutama di lingkungan yang kompleks dan dinamis, masih menjadi tantangan yang signifikan. Pemantauan tradisional, seperti pengawasan oleh pengasuh atau penggunaan kamera CCTV, memiliki keterbatasan dalam hal jangkauan, efektivitas, dan privasi.
Sistem pemantauan otomatis menawarkan solusi yang lebih komprehensif dan terintegrasi, dengan potensi untuk meningkatkan keamanan dan kemandirian pengguna kursi roda.

Sistem dapat membantu pengguna kursi roda untuk menavigasi lingkungan yang kompleks dengan aman: Sistem dapat mendeteksi rintangan, kendaraan, dan bahaya lainnya, dan memberikan peringatan kepada pengguna.
Sistem mengikuti pendamping atau orang di depan dengan melacak pergerakan pendamping atau orang di depan dan mengarahkan kursi roda secara otomatis untuk mengikuti mereka.
Sistem juga memperoleh bantuan saat dibutuhkan dengan mendeteksi situasi darurat, seperti jatuh, dan secara otomatis meminta bantuan.


\section{Permasalahan}
\label{sec:permasalahan}

Dari permasalahan tersebut maka pemantauan manusia pada kursi roda dengan fokus pada kemandirian menghadapi beberapa tantangan utama:

\begin{enumerate}[nolistsep]

      \item Kompleksitas lingkungan: Lingkungan yang dinamis dan tidak terstruktur, seperti trotoar, jalan raya, dan area publik, menghadirkan berbagai rintangan dan bahaya yang sulit diprediksi.

      \item Keterbatasan deteksi tradisional: Metode deteksi tradisional, seperti sensor inframerah atau kamera statis, mungkin tidak dapat mendeteksi semua potensi bahaya secara akurat dan tepat waktu.

      \item Privasi dan etika: Sistem pemantauan harus mempertimbangkan privasi pengguna kursi roda dan menghindari pengumpulan data yang tidak perlu atau sensitif.

\end{enumerate}
\newpage

\section{Tujuan}
\label{sec:Tujuan}

Tujuan dari penelitian ini bertujuan untuk mengembangkan sistem pemantauan manusia pada kursi roda yang efektif, handal, dan berfokus pada privasi. Sistem ini akan memanfaatkan algoritma kecerdasan buatan (AI) canggih untuk:

\begin{enumerate}[nolistsep]

      \item Melacak kursi roda dan pendamping atau orang di depan: Sistem akan menggunakan deteksi objek dan pelacakan untuk mengikuti pergerakan kursi roda dan pendamping atau orang di depan.

      \item Memperkirakan pose pengguna: Sistem akan menggunakan estimasi pose untuk mengidentifikasi posisi bahu dan pinggul pengguna, yang dapat membantu dalam mendeteksi potensi bahaya dan memberikan peringatan.

      \item Mendeteksi bahaya dan memberikan peringatan: Sistem akan menggunakan pengenalan objek dan klasifikasi untuk mendeteksi rintangan, kendaraan, dan bahaya lainnya, dan memberikan peringatan kepada pengguna melalui suara, getaran, atau visual.

      \item Mengontrol kursi roda secara otomatis: Sistem akan menggunakan algoritma kontrol cerdas untuk mengarahkan kursi roda secara otomatis mengikuti pendamping atau orang di depan, dengan mempertimbangkan keamanan dan kenyamanan pengguna.

\end{enumerate}

\section{Batasan Masalah}
\label{sec:batasanmasalah}

Batasan-batasan dari penelitian ini diharapkan memberikan kontribusi signifikan dalam bidang pemantauan manusia pada kursi roda dengan fokus pada kemandirian dan kontrol otomatis, dengan:

\begin{enumerate}[nolistsep]

      \item Mengembangkan algoritma deteksi objek, pelacakan, dan estimasi pose yang akurat dan robust untuk lingkungan yang kompleks.

      \item Merancang sistem kontrol kursi roda otomatis yang aman, nyaman, dan adaptif terhadap berbagai kondisi lingkungan.

      \item Memastikan privasi pengguna kursi roda dengan meminimalkan pengumpulan data sensitif dan menerapkan teknik anonimisasi.

      \item Menyediakan sistem yang mudah digunakan, portabel, dan hemat biaya untuk meningkatkan aksesibilitas bagi pengguna kursi roda.

\end{enumerate}

\section{Sistematika Penulisan}
\label{sec:sistematikapenulisan}

Laporan penelitian tugas akhir ini terbagi menjadi \lipsum[1][1-3] yaitu:

\begin{enumerate}[nolistsep]

      \item \textbf{BAB I Pendahuluan}

            Bab ini berisi \lipsum[2][1-5]

            \vspace{2ex}

      \item \textbf{BAB II Tinjauan Pustaka}

            Bab ini berisi \lipsum[3][1-5]

            \vspace{2ex}

      \item \textbf{BAB III Desain dan Implementasi Sistem}

            Bab ini berisi \lipsum[4][1-5]

            \vspace{2ex}

      \item \textbf{BAB IV Pengujian dan Analisa}

            Bab ini berisi \lipsum[5][1-5]

            \vspace{2ex}

      \item \textbf{BAB V Penutup}

            Bab ini berisi \lipsum[6][1-5]

\end{enumerate}
