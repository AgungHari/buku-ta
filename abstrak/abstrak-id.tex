\begin{center}
  \large\textbf{ABSTRAK}
\end{center}

\addcontentsline{toc}{chapter}{ABSTRAK}

\vspace{2ex}

\begingroup
% Menghilangkan padding
\setlength{\tabcolsep}{0pt}

\noindent
\begin{tabularx}{\textwidth}{l >{\centering}m{2em} X}
  Nama Mahasiswa    & : & \name{}         \\

  Judul Tugas Akhir & : & \tatitle{}      \\

  Pembimbing        & : & 1. \advisor{}   \\
                    &   & 2. \coadvisor{} \\
\end{tabularx}
\endgroup

% Ubah paragraf berikut dengan abstrak dari tugas akhir
Pada penelitian ini kami memantau disabilitas pengguna kursiroda merupakan aspek penting dalam pengembangan teknologi mobilitas. Penelitian ini bertujuan untuk menganalisis algoritma pemantauan manusia yang menggabungkan deteksi bahu dan pinggul menggunakan YOLOv8n-pose pada kursiroda. Metode yang digunakan melibatkan analisis performa algoritma, perbandingan dengan metode lain, dan pengujian di lingkungan nyata. Tujuan penelitian lebih lanjut adalah mengembangkan sistem autopilot yang akan mengikuti pendamping yang berada di depan pengguna kursiroda.

% Ubah kata-kata berikut dengan kata kunci dari tugas akhir
Kata Kunci: Kursiroda, YOLOv8n-pose, Pemantauan Manusia, Deteksi Pose, Algoritma.
