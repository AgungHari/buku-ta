% Atur variabel berikut sesuai namanya

% nama
\newcommand{\name}{Aldifahmi Sihotang}
\newcommand{\authorname}{Sihotang, Aldifahmi}
\newcommand{\nickname}{Aldi}
\newcommand{\advisor}{Dr. Eko Mulyanto Yuniarno, S.T., M.T}
\newcommand{\coadvisor}{Dion Hayu, S.T., M.T}
\newcommand{\examinerone}{}
\newcommand{\examinertwo}{}
\newcommand{\examinerthree}{}
\newcommand{\headofdepartment}{}

% identitas
\newcommand{\nrp}{0721 18 4000 0039}
\newcommand{\advisornip}{19680601 1995121 1 009}
\newcommand{\coadvisornip}{18560710 194301 1 001}
\newcommand{\examineronenip}{}
\newcommand{\examinertwonip}{}
\newcommand{\examinerthreenip}{}
\newcommand{\headofdepartmentnip}{}

% judul
\newcommand{\tatitle}{pengembangan Kursi Roda Otonom dengan ESP32-CAM Berbasis YOLOv8}
\newcommand{\engtatitle}{\emph{ANALYSIS OF HUMAN MONITORING ALGORITHM IN WHEELCHAIRS WITH YoloV8}}

% tempat
\newcommand{\place}{Surabaya}

% jurusan
\newcommand{\studyprogram}{Teknik Komputer}
\newcommand{\engstudyprogram}{Computer Engineering}

% fakultas
\newcommand{\faculty}{Teknologi Elektro dan Informatika Cerdas}
\newcommand{\engfaculty}{Intelligent Electrical and Informatics Technology}

% singkatan fakultas
\newcommand{\facultyshort}{FTEIC}
\newcommand{\engfacultyshort}{F-ELECTICS}

% departemen
\newcommand{\department}{Teknik Komputer}
\newcommand{\engdepartment}{Computer Engineering}

% kode mata kuliah
\newcommand{\coursecode}{TD123456}
